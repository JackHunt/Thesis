The model of the proposed approach is of a siamese topology, where for a given stereo 
pair each of the left and right frames are passed to a separate network. From end to 
end, the model contains for each of the siamese networks a ResNet-50~\cite{RESNET}, 
Linear Layers, Nonlinear Activations, a mapping from Lie Algebra to Lie Group (for pose),
a GPLVM followed by an Inverse Discrete Cosine Transform, a Raycaster and an information 
theoretic loss. 

The purpose of the ResNet-50 components is to extract pertinent features from the stereo 
pair that are descriptive of the the object of interest in relation to the scene. For each 
of the networks, the output of the ResNet-50 network is fed into two subnetworks consisting 
of linear transforms and nonlinear activations. The purpose of these subnetworks is to regress 
a latent space point for object shape, and a 6DoF pose parameter vector for the object pose. 
Following these two subnetworks is the aforementioned GPLVM and Lie Algebra mapping. The 
GPLVM generates a posterior mean over shape for a given latent space point, whilst the 
Lie Algebra mapping generates an \( \mathbb{SE}(3) \) transform. The IDCT decompresses 
the posterior mean output of the GPLVM to generate a valid SDF. Both the resultant SDF and 
\( \mathbb{SE}(3) \) transform are passed to the Raycasting module, which generates a rendering 
for the candidate shape and pose. Finally, this rendering with the initial detection is passed 
to the loss layer which computes appearance statistics for the detection region and the rendering 
which are used to quantify loss. The topology of the proposed model is outlined in Figure
~\ref{}
FIGURE HERE