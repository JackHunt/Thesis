\message{ !name(object_reconstruction.tex)}
\message{ !name(object_reconstruction.tex) !offset(258) }

as follows.
\begin{equation}
  \label{eqn:probobj_appearanc_marginal}
  \begin{split}
    % Line 1.
    P(\Phi, \Omega, p, L) & =
    \int_{-\infty}^{\infty} \Bigg[ 
    \prod_{\psi \in \mathbf{\Psi}} P(\Phi \given u_{\psi})
    \prod_{s, s' \in \mathcal{S}}P(u_{\psi} \given \Omega, p, L_{s, s'})
    P(L_{s, s'} \given \Omega, p) P(L_{s, s'})P(p)P(\Omega) \Bigg] \intd{u} \\
    % Line 2.
    & = \prod_{\psi \in \mathbf{\Psi}} 
    \int_{-\infty}^{\infty} \Bigg[ P(\Phi \given u_{\psi})
    \prod_{s, s' \in \mathcal{S}}P(u_{\psi} \given \Omega, p, L_{s, s'})
    P(L_{s, s'} \given \Omega, p) P(L_{s, s'})P(p)P(\Omega) \Bigg] \intd{u} \\
    & = \prod_{s, s' \in \mathcal{S}} P(L_{s, s'} \given \Omega, p)
    P(L_{s, s'})P(p)P(\Omega)P(\mathbf{\Phi})
  \end{split}
\end{equation}
Note that the Appearance Posterior Volume outlined in Section 
\ref{subsec:probobj_vol_appearance_model} is reintroduced later in this work in
Section \ref{} for the purposes of subvolume alignment and determining the
subset of Voxels $\mathbf{\Phi} \subset \mathbf{\Psi}$ that determine the target
object shape.

Further details pertaining to the inference procedure for the per-subvolume
deformations is provided in Section \ref{sec:probobj_model_correction}.

\section{Online Model Correction}
\label{sec:probobj_model_correction}
The tracking consistency constraints denoted by the variables $L_{s, s'}$ such
that $s, s' \in \mathcal{S}$ with $\mathcal{S}$ being the set of overlapping
subvolume pairs $s, s'$ in the Probabilistic Graphical Models given by Figures
\ref{fig:probobj_pgm1} and \ref{fig:probobj_pgm2} can be enforced in terms of
minimising the disparity between each pair of adjacent subvolumes. The effect of
this minimsation being that consistency in the Pose Estimation phase of the
Pipeline outlined in Figure \ref{fig:probobj_pipeline_diagram} is enforced. The
objective of this procedure is to infer a robust and consistent deformation
transformation for the subvolume pair.

\subsection{Alignment MAP Estimate}
\label{subsec:probobj_alignment_map}
Referring back to the joint distribution of Equation
\ref{eqn:probobj_simplified_joint}, to achieve the aforementioned minimisation
of disparity between overlapping subvolumes, a Maximum a Posteriori (MAP)
estimate is desirable. As such, a MAP estimate over $L_{s, s'}$ in Equation
\ref{eqn:probobj_simplified_joint} for a given subvolume pair $s, s'$ may be
derived as follows.
\begin{equation}
  \label{eqn:probobj_map_estimate}
  \begin{split}
    % Line 1.
    P(\Omega, p | L_{s, s'}) & \propto \frac{P(L_{s, s'} | \Omega, p) 
    P(\Omega | p)P(p)P(L_{s, s'})}
    {\displaystyle\int_{-\infty}^{\infty} P(L_{s, s'} \given \Omega, p)
    \intd{L_{s, s'}}} P(\Phi) \\
    % Line 2.
    & \propto P(L_{s, s'} | \Omega, p) P(\Omega | p)P(p)P(L_{s, s'})P(\Phi) \\
    % Line 3.
    & \propto P(L_{s, s'} | \Omega, p)P(L_{s, s'})P(\Phi)
  \end{split}
\end{equation}
Note that in the third step of Equation \ref{eqn:probobj_map_estimate} the
distributions $P(\Omega | p)$ and $P(p)$ are taken to be Uniform and as such may
be omitted whilst retaining proportionality. The distribution $P(L_{s, s'})$ is
Conjugate to $P(L_{s, s'} | \Omega, p)$ and is of the form of a Multivariate
Gaussian Distribution $\mathcal{N}(\mathbf{1} \given \mathbf{0})$ over the
$\mathbb{SE}(3)$ deformation parameters. The choice of such a Prior Distribution
is motivated by the assumption that motion between consecutive frames is minor,
thus the given Prior will have the effect of constraining the $\mathbb{SE}(3)$
transformation as such. The Prior distribution $P(\Phi)$ serves as a
\textit{Surface Prior} to mitigate the effect of noise introduced in to the TSDF
volumes. The form of $P(\Phi)$ shall be discussed in Section \ref{}.

The rationale of Equation \ref{eqn:probobj_map_estimate} is that the deformation
$L_{s, s'}$ applied to the subvolume $s$ maximises the Posterior Probability of
observing the current pose $p$ given the current RGBD frame $\Omega$ by reducing
the variance of the result of the Pose Estimation phase of the Pipeline. As such,
global tracking variance (quantified by the proportion of outliers in the result
of the ICP component of the Pipeline) is reduced by enforcing local consistency,
also improving global consistency and thus the quality of the resultant

\message{ !name(object_reconstruction.tex) !offset(-86) }
