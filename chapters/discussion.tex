~\label{chap:discussion}
\begin{chapterabstract}
This chapter reviews the proposed algorithms in this work within the context 
of the initially outlined research objectives of this thesis.
\end{chapterabstract}

\section{Summary}
~\label{sec:discussion_summary}
This work has outlined novel approaches to various challenges in 3D computer vision, 
including handling dynamics in dense SLAM, reconstructing 3D objects and predicting both 
shape and pose. This chapter evaluates the methods, contributions and outcomes of these 
approaches, with the research objectives outlined in Section~\ref{sec:intro_aims_structure} 
as a point of reference.

This chapter proceeds as follows; first, Section~\ref{subsec:discussion_moseg} provides a 
summary of the dynamic SLAM and motion segmentation approach outlined in Chapter~\ref{chap:moseg}.
Following this, Section~\ref{subsec:discussion_probobj} provides a similar evaluation of the 
approach to object reconstruction outlined in Chapter~\ref{chap:probobj}. Section
~\ref{subsec:discussion_spp} evaluates the approach taken to shape and pose prediction in 
Chapter~\ref{chap:spp}.

Following the evaluations given in Sections~\ref{subsec:discussion_moseg},~\ref{subsec:discussion_probobj} 
and~\ref{chap:spp}, consideration is given to the limitations of the approaches of this work and potential 
future directions in Section~\ref{sec:discussion_limitations}. Finally, a high level conclusion and closing 
remarks are given in Section~\ref{sec:discussion_conclusions}.

\subsection{Real Time Motion Segmentation for Dense Volumetric Fusion}
~\label{subsec:discussion_moseg}
One of the central research objectives outlined in Section~\ref{sec:intro_aims_structure} 
is the development of an algorithm which facilitates the dense reconstruction of dynamic 
environments. As highlighted in Sections~\ref{sec:intro_scene_recon} and
~\ref{sec:lit_review_dynamic}, this has remained a challenging problem in 3D computer vision, 
due to the reliance of prominent pose estimation techniques on reliant point correspondences 
between frames. The novel approach taken in Chapter~\ref{chap:moseg} mitigates the restrictions 
incurred by this dependence by actively excluding dynamic scene components from the pose 
estimation phase of the pipeline, as shown in Figure~\ref{figure:moseg_pipeline}.

As demonstrated in Sections~\ref{sec:moseg_qualitative} and~\ref{sec:moseg_quantitative}, the 
approach taken demonstrates improvements in pose estimation quality over an open implementation
~\cite{Prisacariu2014} of the standard KinectFusion~\cite{Newcombe2011} pipeline when evaluated 
on the \textit{Dynamic Objects} subset of the TUM RGBD dataset~\cite{Sturm2012}. This improved 
performance is evident in Tables~\ref{table:moseg_ate} and~\ref{table:moseg_rte}, and 
Figures~\ref{figure:moseg_ate} and~\ref{figure:moseg_rte}. The improvement in the ability to 
accurately track sensor pose during reconstruction fulfils the research objective of designing 
an algorithm that provides an improvement in pose estimation versus static dense SLAM in dynamic 
scenes.

Additionally, it is evident from Section~\ref{sec:moseg_qualitative} that the proposed approach 
yields high quality reconstructions in dynamic environments that are comparable to their static 
counterparts when there is no motion in the scene. Again, this result is directly satisfying of 
the research objective to design a system that is capable of providing comparable quality 
reconstructions in previously troublesome environments.

Section~\ref{sec:moseg_semantic} demonstrates how the dynamic segmentation ability of the proposed 
approach may be leveraged for semantic purposes. It is shown that the use of the dynamics information 
may be used to indicate to the system an object of interest, such that 3D features may be extracted 
and used for training of, and prediction with simple classifiers in an interactive manner. In Section
~\ref{sec:intro_aims_structure} potential use of dynamics for semantic purposes is outlined as an 
additional research objective.

%-> Specific contributions.
%--> Novel expansion to classic KF pipeline
%--> Dual represenation, adaptive mu
The demonstrated improvements in pose estimation and the ability to perform simple semantic 
understanding in dynamic scenes are facilitated by the following central contributions of 
Chapter~\ref{chap:moseg}. First is the introduction of the novel \textit{dual representation} of 
the scene, in which a \textit{stable} version of the scene is maintained as both the resultant 
reconstruction and the source of depth map to which live frames are registered. The 
\textit{dynamic} scene representation has all observed data points integrated. This dual 
representation allows for the separation and thus segmentation of dynamic, moving scene components 
from their stable counterparts, thus allowing the pose estimation phase to use a reliable, non 
corrupted scene model.

In addition to the dual scene representation, the system outlined in Chapter~\ref{chap:moseg} 
introduces a novel online adaption schema for the TSDF truncation region. This online adaption 
of the truncation region allows for live integration and removal of surface data in the dynamic 
model by facilitating real time space carving such that changes in the scene are reflected 
instantly in the dynamic model. This technique combined with the dual representation is the 
basis of the proposed approach.

The system and results outlined in Chapter~\ref{chap:moseg} have the potential to impact 
significantly on real world applications of 3D vision systems. Due to the approach utilising 
volumetric representations versus less scalable alternatives, there is potential for application 
in large scale robotics. Such applications are ordinarily inhibited by the static nature of 
dense SLAM approaches. Additionally, the ability to operate in dynamic environments when coupled 
with the ability to exploit dynamics has the potential to greatly impact on the fields of semantic 
dense SLAM.

\subsection{Probabilistic Object Reconstruction with Online Drift Correction}
~\label{subsec:discussion_probobj}
%-> Dense reconstruction of objects
%--> Globally consistent
%---> As shown by results...
%---> Comparison to SoTA
%---> Previously problematic due to...
%----> As shown by results...
%--> Without known pose and with commodity hardware
%---> Initial aim of ease of use and applicability

%-> Impact of results.
%--> Globally consistent, dense reconstructions
%--> Runs without specialist hardware (unlike many current systems)
%--> Potential applications in data profcurement for deep learning etc

\subsection{Stereo Shape and Pose Regression}
~\label{subsec:discussion_spp}

\section{Future Work and Limitations}
~\label{sec:discussion_limitations}
%-> Though the contributions of this work..., future work...

%-> Moseg
%--> Limitations
%---> Currently maintains two full SDF's; this can be optimised
%---> Single "moving thing" class; can add more semantics
%--> Future work
%---> Optimise pipeline for better space complexity
%---> Introduce semantics throughout; multi moving things etc
%---> Expand to dynamic semantic slam system; a la spaint

%-> Object reconstruction.
%--> Limitations
%---> Purely observation driven; could better leverage stochastic framework
%---> Requires manual indication of object of interest
%--> Future work
%---> Expand probability framework to directly estimate missing geometry
%---> Exploit advances in saliency to remove manual indication requirement
%---> Multiple objects?

%-> General integration of moseg and object recon for super semantic slam

\section{Conclusions}
~\label{sec:discussion_conclusions}