~\label{chap:discussion}
\begin{chapterabstract}
This chapter reviews the proposed algorithms in this work within the context 
of the initially outlined research objectives of this thesis.
\end{chapterabstract}

\section{Summary}
~\label{sec:discussion_summary}
%-> This thesis has developed novel algo's for...
%-> Progress towards applications...
%-> This chapter proceeds...

Chapters~\ref{}

\subsection{Real Time Motion Segmentation for Dense Volumetric Fusion}
~\label{subsec:discussion_moseg}
%-> Dense reconstruction with dynamics
%--> Aim to have KF like pipeline with dynamics
%---> Previously prohibitive due to...
%---> Results show...
%--> With RT performance
%--> TODO: measure this
%--> With comparable reconstruction quality
%--> With improved pose estimation

%-> Identification/segmentation of moving objects
%--> Demonstrate potential for semantic use.

%-> Specific contributions.
%--> Novel expansion to classic KF pipeline
%--> Dual represenation, adaptive mu

%-> Impact of results
%--> Larger scale reconstructions with dynamics possible (vs dynamicfusion)
%---> Using volumetric representation and all it's advantages (versus non vol)
%---> Potential for application in larger, troublesome environments
%----> Industrial robotics

\subsection{Probabilistic Object Reconstruction with Online Drift Correction}
~\label{subsec:discussion_probobj}
%-> Dense reconstruction of objects
%--> Globally consistent
%---> As shown by results...
%---> Comparison to SoTA
%---> Previously problematic due to...
%----> As shown by results...
%--> Without known pose and with commodity hardware
%---> Initial aim of ease of use and applicability

%-> Impact of results.
%--> Globally consistent, dense reconstructions
%--> Runs without specialist hardware (unlike many current systems)
%--> Potential applications in data profcurement for deep learning etc

\subsection{Stereo Shape and Pose Regression}
~\label{subsec:discussion_spp}

\section{Future Work and Limitations}
~\label{sec:discussion_limitations}
%-> Though the contributions of this work..., future work...

%-> Moseg
%--> Limitations
%---> Currently maintains two full SDF's; this can be optimised
%---> Single "moving thing" class; can add more semantics
%--> Future work
%---> Optimise pipeline for better space complexity
%---> Introduce semantics throughout; multi moving things etc
%---> Expand to dynamic semantic slam system; a la spaint

%-> Object reconstruction.
%--> Limitations
%---> Purely observation driven; could better leverage stochastic framework
%---> Requires manual indication of object of interest
%--> Future work
%---> Expand probability framework to directly estimate missing geometry
%---> Exploit advances in saliency to remove manual indication requirement
%---> Multiple objects?

%-> General integration of moseg and object recon for super semantic slam

\section{Conclusions}
~\label{sec:discussion_conclusions}