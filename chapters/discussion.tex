~\label{chap:discussion}
\begin{chapterabstract}
This chapter reviews the proposed algorithms in this work within the context 
of the initially outlined research objectives of this thesis.
\end{chapterabstract}
% TODO: Expand this abstract.

\section{Summary}
~\label{sec:discussion_summary}
This work has outlined novel approaches to various challenges in 3D computer vision, 
including handling dynamics in dense SLAM, reconstructing 3D objects and predicting both 
shape and pose. This chapter evaluates the methods, contributions and outcomes of these 
approaches, with the research objectives outlined in Section~\ref{sec:intro_aims_structure} 
as a point of reference.

This chapter proceeds as follows; first, Section~\ref{subsec:discussion_moseg} provides a 
summary of the dynamic SLAM and motion segmentation approach outlined in Chapter~\ref{chap:moseg}.
Following this, Section~\ref{subsec:discussion_probobj} provides a similar evaluation of the 
approach to object reconstruction outlined in Chapter~\ref{chap:probobj}. Section
~\ref{subsec:discussion_spp} evaluates the approach taken to shape and pose prediction in 
Chapter~\ref{chap:spp}.

Following the evaluations given in Sections~\ref{subsec:discussion_moseg},~\ref{subsec:discussion_probobj} 
and~\ref{sec:spp_discussion}, consideration is given to the limitations of the approaches of this work and 
potential future directions in Section~\ref{sec:discussion_limitations}. Finally, a high level conclusion and 
closing remarks are given in Section~\ref{sec:discussion_conclusions}.

\subsection{Real Time Motion Segmentation for Dense Volumetric Fusion}
~\label{subsec:discussion_moseg}
One of the central research objectives outlined in Section~\ref{sec:intro_aims_structure} 
is the development of an algorithm which facilitates the dense reconstruction of dynamic 
environments. As highlighted in Sections~\ref{sec:intro_scene_recon} and
~\ref{sec:lit_review_dynamic}, this has remained a challenging problem in 3D computer vision, 
due to the reliance of prominent pose estimation techniques on reliant point correspondences 
between frames. The novel approach taken in Chapter~\ref{chap:moseg} mitigates the restrictions 
incurred by this dependence by actively excluding dynamic scene components from the pose 
estimation phase of the pipeline, as shown in Figure~\ref{figure:moseg_pipeline}.

As demonstrated in Sections~\ref{sec:moseg_qualitative} and~\ref{sec:moseg_quantitative}, the 
approach taken shows improvements in pose estimation quality over an open implementation
~\cite{Prisacariu2014} of the standard KinectFusion~\cite{Newcombe2011} pipeline when evaluated 
on the \textit{Dynamic Objects} subset of the TUM RGBD dataset~\cite{Sturm2012}. This improved 
performance is evident in Tables~\ref{table:moseg_ate} and~\ref{table:moseg_rte}, and 
Figures~\ref{figure:moseg_ate} and~\ref{figure:moseg_rte}. The improvement in the ability to 
accurately track sensor pose during reconstruction fulfils the research objective of designing 
an algorithm that provides an improvement in pose estimation versus static dense SLAM in dynamic 
scenes.

Additionally, it is evident from Section~\ref{sec:moseg_qualitative} that the proposed approach 
yields high quality reconstructions in dynamic environments that are comparable to their static 
counterparts when there is no motion in the scene. Again, this result is directly satisfying of 
the research objective to design a system that is capable of providing comparable quality 
reconstructions in previously troublesome environments.

Section~\ref{sec:moseg_semantic} demonstrates how the dynamic segmentation ability of the proposed 
approach may be leveraged for semantic purposes. It is shown that the use of the dynamics information 
may be used to indicate to the system an object of interest, such that 3D features may be extracted 
and used for training of, and prediction with simple classifiers in an interactive manner. In Section
~\ref{sec:intro_aims_structure} potential use of dynamics for semantic purposes is outlined as an 
additional research objective.

The demonstrated improvements in pose estimation and the ability to perform simple semantic 
understanding in dynamic scenes are facilitated by the following central contributions of 
Chapter~\ref{chap:moseg}. First is the introduction of the novel dual representation of 
the scene, in which a stable version of the scene is maintained as both the resultant 
reconstruction and the source of depth map to which live frames are registered. The 
dynamic scene representation has all observed data points integrated. This dual 
representation allows for the separation and thus segmentation of dynamic, moving scene components 
from their stable counterparts, thus allowing the pose estimation phase to use a reliable, non 
corrupted scene model.

In addition to the dual scene representation, the system outlined in Chapter~\ref{chap:moseg} 
introduces a novel online adaption schema for the TSDF truncation region. This online adaption 
of the truncation region allows for live integration and removal of surface data in the dynamic 
model by facilitating real time space carving such that changes in the scene are reflected 
instantly in the dynamic model. This technique combined with the dual representation is the 
basis of the proposed approach.

The system and results outlined in Chapter~\ref{chap:moseg} have the potential to impact 
significantly on real world applications of 3D vision systems. Due to the approach utilising 
volumetric representations versus less scalable alternatives, there is potential for application 
in large scale robotics. Such applications are ordinarily inhibited by the static nature of 
dense SLAM approaches. Additionally, the ability to operate in dynamic environments when coupled 
with the ability to exploit dynamics has the potential to greatly impact on the fields of semantic 
dense SLAM\@.

\subsection{Probabilistic Object Reconstruction with Online Drift Correction}
~\label{subsec:discussion_probobj}
The second major research objective outlined in Section~\ref{sec:intro_aims_structure} is the 
development of a system that allows for the reconstruction of arbitrary objects in a globally 
consistent manner. As highlighted in Sections~\ref{sec:intro_object_recon} and
~\ref{sec:lit_review_obj_recon}, object centric dense reconstruction is a markedly difficult 
problem compared to it's scene scale counterpart. A prominent technical challenge in object 
reconstruction is the enforcement of object consistency when there are erroneous pose estimation 
results. The system outlined in Chapter~\ref{chap:probobj} mitigates object inconsistencies by 
the use of a novel object reconstruction pipeline that includes an online correction procedure.

The system outlined in Chapter~\ref{chap:probobj} when compared to the state-of-the-art object 
reconstruction method of \textit{Ren et al}~\cite{Ren2013}, demonstrates a vast improvement in 
efficacy for general object reconstruction. The latter system was unable to reconstruct the test 
sequences used in Sections~\ref{sec:probobj_qualitative} and~\ref{sec:probobj_quantitative}. 

Furthermore, it is shown in Table~\ref{table:probobj_hauss} and Figure~\ref{figure:probobj_hauss} 
of Section~\ref{sec:probobj_quantitative} that the approach presented in this work is capable of 
yielding high quality reconstructions, relative to manually extracted reconstructions obtained 
with a standard, scene scale dense SLAM approach. As outlined in Section
~\ref{sec:intro_aims_structure}, a principal requirement of the object reconstruction research 
objective is to develop an approach that provides high quality reconstructions, utilising only 
observations belonging to the object of interest.

Whereas many dedicated object reconstruction systems rely on either known poses or poses for which 
there is a strong prior (such as an object on a turntable in front of a laser scanner), the approach 
proposed in this work is able to perform pose estimation online for arbitrary trajectories. In 
Section~\ref{sec:intro_aims_structure} one of the requirements of the central research objective of 
the work of Chapter~\ref{chap:probobj} is that the system is capable of operating with no a-priori
known trajectory. This competency may be observed in Table~\ref{table:probobj_ate} and Figure
~\ref{figure:probobj_comp_itm} of Section~\ref{sec:probobj_quantitative}. This competency allows 
the system to be used with commodity equipment, thus facilitating ease of use.

The ability of the proposed object reconstruction system to produce globally consistent reconstructions 
is demonstrated in Section~\ref{sec:probobj_qualitative} and is one of the key research objectives of the 
work of Chapter~\ref{chap:probobj}. The potential impact of a system that can be used to easily obtain such 
reconstructions is particularly evident for fields that are heavily data dependent. Such a field is machine 
learning, where as outlined in Section~\ref{sec:intro_object_recon}, there is an abundance of real world 2D 
image data available for the learning of vision tasks, but not a comparable amount for 3D geometry.

\subsection{Stereo Shape and Pose Regression}
~\label{subsec:discussion_spp}

\section{Future Work}
~\label{sec:discussion_limitations}
The system of Chapter~\ref{chap:moseg} provides a strong foundation for further research 
into both dense SLAM for dynamic environments and semantic SLAM\@. However, there are 
potential directions of subsequent research that follow on directly from the approach of 
this work. Firstly, at present the system presented in this work currently maintains two full 
TSDF volumes; there is a volume for the static scene representation and a volume for the dynamic 
scene representation. An improvement that could be introduced in a later iteration of this work 
is the use of a sparse representation of the dynamic scene volume which would improve the space 
complexity of the system. 

A second potential improvement to the work of Chapter~\ref{chap:moseg} is the expansion of the 
motion segmentation itself to be semantically aware. At present, the system is capable of 
determining which elements of the world are undergoing motion and is able to leverage this 
information to improve pose estimation and thus the quality of the resultant reconstruction. 
With further semantic capabilities at the lower level, a revised system could have the ability 
to separate observed motion into instances of dynamic objects.

The object reconstruction system presented in Chapter~\ref{chap:probobj} provides a simple means 
to obtain geometrically consistent reconstructions of 3D objects. Though the proposed system
demonstrates an improvement over comparative methods, it still requires the end user to indicate 
which object is to be reconstructed. An alternative approach would be to leverage some of the 
advancements in salient object detection~\cite{Borji2014} to allow for automation of the process.
% TODO: salient region ref

The probabilistic framework on which the system is based could additionally be extended in a later 
iteration of the work to estimate the geometry of any missing sections of the surface of the object 
being reconstructed. Such an extension would further increase the consistency of the reconstructions 
in the case where full coverage with the sensor has not been achieved.

Another potential line of research following on from the contributions outlined is the extension to 
the multiple object case. The ability to track and reconstruct multiple objects in the sensor view 
combined with the potential salient object detection would greatly enhance the efficiency of 3D data 
collection.

\section{Closing Remarks}
~\label{sec:discussion_conclusions}
This work has introduced novel approaches to a range of nontrivial research problems in 3D computer 
vision. The contributions of this work span the extension of the traditional dense SLAM pipeline to 
the dynamic case, the consistent reconstruction of arbitrary objects and the prediction of real world 
objects for which only a single view pair is available. The remit of this work has direct relevance to 
the increasingly automated world of machine perception and intelligence.

It is the authors hope that this relevance is useful to practitioners in the field in their endeavours 
to continue the rapid technological advancement of our society.