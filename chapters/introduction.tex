% Introduction of Introduction
% -- Increased compute changed how people interact with computers
% ---- Games to social media, medical imaging to service robotics
% ---- Immersive virtual worlds - VR
% ---- Virtual to real world - AR
% -- The above is enriched by vast amounts of data and modern ML
% ---- Semantics, detections etc
% -- 3D sensing equipment allows for geometric modelling of real world
% ---- Reconstructions, 3D motion estimation, pose tracking
% ---- 3D sensing combined with ML allows for rich scene understanding; link back to VR/AR
% ---- 3D data not in abundance, unlike 2D images; harder to train task specific systems
Much has changed since the introduction of the IBM PC platform in 1981, with regards to the 
uses of the Personal Computer and the way in which humans interact with them. Vast increases 
in computational power have seen the microcomputer evolve beyond a tool with which one predominantly 
collated data with spreadsheets (``VisiCalc'' is widely accepted to be the first \textit{killer app}
for the PC). At present, personal computing is ubiquitous, providing the means for activities ranging 
from playing visually realistic 3D games, to sharing life events online with friends and family via 
so called ``Social Media'' platforms. Advances in graphics processing power have lead to the introduction 
of many technologies that both the public sector and private enterprises have come to depend on, from 
sophisticated medical image analysis, to robotic production lines.

The aforementioned vast power of the modern GPU (Graphical Processing Unit) has lead to the development 
of new ways for humans to interact with computers, and for computers to interact with the world. Advances 
in VR (Virtual Reality) allow humans to experience digital worlds in a fully immersive manner, with 
applications ranging from gaming, to experimental therapeutic treatment of Paranoid Schizophrenia. 
Whereas AR (Augmented Reality) allows a person to supplement their perception of the tangible world with 
elements of the digital, by viewing the world through a computer system that percieves and adds synthetic 
components. Common applications of AR include digital interior design and smartphone games.

Many of the aforementioned advancements in the applications of and interactions with modern computer 
systems are driven by Artificial Intelligence (AI). Modern AI provides much of the semantic and contextual 
information required to make meaningful inferences over the state of the world, as observed by a sensor 
(such as a camera) attached to such a computer system. Much advancement has been made in recent years on 
the tasks of object detection and semantic understanding in standard, 2D images. However, there are many 
technical challenges that must be overcome before such efficacy on these tasks is reached for the 3D 
case.



% Topic overview.
% -- 3D machine perception
% ---- optical flow
% ---- laser range sensing
% ---- navigation and mapping with sparse point clouds
% ---- stereo and rgbd
% ---- multiple view geometry
% ---- SfM, SLAM, scene flow
% -- Semantics
% ---- object recognition and segmentation
% ---- user interaction
% -- Amalgamation
% ---- applications in VR/AR
% ---- ad-hoc ML (3d data procurement etc)
% -- Current challenges
% ---- fragility in complex, real environments (dynamics etc)
% ---- lack of real world data for learning of 3d
% ---- SLAM limited to observations only; may not see whole thing; may not be practical etc

% Elaborate on current challenges.
% -- Dynamics in SLAM
% ---- for SLAM in real environments (working environments such as warehouses), no consistent model
% ---- fragility leads to problems; complete failure to inaccuracies
% ---- reliance on point correspondences; efficacious for static, fails for dynamic
% ---- no dynamics = limited perception capability; needed for "real" environments
% ---- dynamics provide semantic information
% -- Object centric SLAM/SfM
% ---- preliminary work limited on simultaneous TaM; offline or known pose
% ---- fundamentally a difficult problem; less to work with
% ---- geometric inconsistencies more pronounced effect; bad training data etc
% -- Data driven pose/shape regression
% ---- complex environments may not allow long enough to fuse model
% ---- larger scale makes dense reconstruction infeasable
% ---- limited observability; not able to fully cover object
% ---- eliminate many lower level "tricks" to make reconstruction work

% Research questions.
% -- Can we reconstruct densely in dynamic scenes?
% ---- real time?
% ---- comparable reconstruction quality to state of art?
% ---- can we improve tracking in these scenarios?
% -- Can we determine what is dynamic in a scene?
% ---- Can we leverage this information in some semantic way (link back to AR applications/data collection)?
% -- Can we obtain consistent reconstructions of arbitrary objects?
% ---- comparable reconstruction quality to state of the art scene based?
% ---- with commodity hardware?; wider applicability
% ---- without known pose?
% ---- 
% -- Can we infer scene properties where traditional reconstruction not possible?
% ---- shape and pose?
% ---- without requiring temporal consistency?; avert tracking errors
% ---- 

% Thesis roadmap.
% -- Literature review.
% ---- TaM/SLAM
% ---- Semantics/Semantic SLAM
% ---- Dynamics, moseg and optical flow
% ---- Object reconstruction
% ---- Shape and pose regression
% -- "Real Time Motion Segmentation for Dense Volumetric Fusion"
% ---- Outline preliminaries; standard dense SLAM etc
% ---- Approach to dynamic dense SLAM
% ---- Segmentation of dynamics; which questions adressed?
% ---- Rudimentary example of semantics; which questions adressed?
% ---- Outline experiments
% ---- Results showing...
% -- "Probabilistic Object Reconstruction with Online Drift Correction"
% ---- Outline approach; which questions adressed?
% ---- Outline experiments
% ---- Outline results
% -- "Stereo Shape and Pose Regression"
% ---- Outline approach; which questions adressed?
% ---- Outline experiments
% ---- Outline results