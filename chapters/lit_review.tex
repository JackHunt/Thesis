~\label{chap:lit_review}
\begin{chapterabstract}
This chapter provides a comprehensive survey of pertinent literature in the fields 
of Tracking and Mapping, Semantic SLAM, Dynamics in 3D Vision, Object Reconstruction 
and the prediction of Pose and Shape.
\end{chapterabstract}

%% Tracking and Mapping.
\section{Tracking and Mapping}
~\label{sec:lit_review_tam}
There has been much research in the field of Tracking and Mapping (TAM) in recent 
years, with many large scale works being driven by the availability of once 
costly depth sensing equipment. The availability of such equipment combined 
with the ever increasingly parallel nature of modern Graphical Processing Units (GPU) has 
seen the field advance greatly beyond the seminal but compute resource limited works of it's 
infancy. This advancement is most predominant within the dense SLAM literature. 
This section shall first explore the earlier, fundamental works of this area of research, 
followed by an assessment of the current state of the art in dense SLAM\@.

\textit{Besl \& McKay}~\cite{Besl1992} introduced their seminal work on 3D shape 
registration in 1992, providing a method to estimate the full Six Degrees of 
Freedom (DoF) pose between 3D point sets. The authors present an Iterative Closest 
Point Algorithm that consists of three operations per iteration; computation 
of closest point, computation of a 6DoF transformation and application of 
the transformation. The authors present a proof of convergence based on that 
of least squares minimisation, however there must be sufficiently complex 
geometry present in the structure of the data to converge to a meaningful 
transformation. The algorithm introduced is commonly known as Iterative Closes Points (ICP).

Complementary to the aforementioned works of \textit{Besl \& McKay}~\cite{Besl1992} 
in the foundational aspects of Dense SLAM is that of \textit{Curless \& Levoy} 
~\cite{Curless1996}, 1996. The authors present an early volumetric integration 
framework for the reconstruction of shapes from range data obtained from a 
sensor such as a laser scanner. The authors introduce the Signed Distance Function (SDF), 
a volumetric, implicit shape representation in which entries are cumulatively updated 
in a weighted manner. Once observations have been integrated in to the SDF volume, an 
isosurface representation of the shape is extracted by a Marching Cubes~\cite{Lorensen1987} 
procedure. Though the approach may lead to gaps in the resultant model, the authors 
mitigate this by introducing a surface tessellation step.

\textit{Klein \& Murray}~\cite{Klein2007}, in 2007 introduced the seminal, sparse TAM work 
\textit{PTAM\: Parallel Tracking and Mapping}. The authors present an approach to the parallel 
estimation of camera pose and sparse scene mapping, making use of early parallel CPU hardware. 
The proposed scene representation is a map, consisting of point based features and multiple resolution 
key-frames. Pose estimation is achieved by the minimisation of the projection error between map points and 
the current live frame, following a coarse-to-fine strategy. The authors report per-frame running times of 
\( \approx 20ms \), irrespective of increases in map size. However, it is reported in the work that failure 
cases exist; blurring of frames can impede feature detection, as can a lack of rigid geometry in the scene (due 
to the systems dependence on corner based features).

Later work by \textit{Zhou et al}~\cite{Zhou2008} in 2008 introduces an alternative 
shape representation to that of \textit{Curless \& Levoy}~\cite{Curless1996}, based 
on the spatial KD-Tree data structure and a highly data parallel Breadth First 
Search (BFS) construction algorithm. The level of parallelism introduced allows for 
application to problems that require real time performance. The authors provide 
examples of use in ray tracing~\cite{Purcell2002} and photon mapping~\cite{Kajiya1986}.

In the same year, \textit{Censi}~\cite{Censi2008} introduced Point-to-Line ICP, a 
variant of the ICP algorithm introduced in the work of \textit{Besl \& McKay} 
~\cite{Besl1992}. The presented approach utilises a point-to-line metric rather than 
a point-to-point metric and has a closed form solution in the planar case. For the non 
planar case the presented approach achieves quadratic convergence in a finite number of 
steps, utilising a normal weighting and a Lagrangian optimisation scheme. However, it is 
highlighted that prior to the optimisation procedure it is necessary to trim outliers 
from the point data sets as an additional preprocessing procedure.

In 2011, \textit{Newcombe et al}~\cite{Newcombe2011_2} introduced an approach to dense TAM 
using monocular RGB images. Unlike the work of \textit{Klein \& Murray}~\cite{Klein2007}, the 
authors do not make use of features extracted from the scene, but rather dense, textured depth 
maps. However, similar to the approach of \textit{Klein \& Murray}~\cite{Klein2007}, the authors 
generate map components on a keyframe basis, which are used for the pose estimation phase within 
a photometric loss against a dense model. The authors report real time performance with commodity 
GPU acceleration. However, due to the reliance on a monocular RGB source, the authors report failure 
cases in the presence of illumination changes in the scene.

Culminating much of the aforementioned work, in 2011 \textit{Newcombe et al} 
~\cite{Newcombe2011} also introduced the seminal \textit{KinectFusion} pipeline, allowing 
for real time mapping of indoor scenes with the Kinect RGBD sensor from Microsoft. The 
authors utilise a sparser form of the SDF structure introduced by \textit{Curless \& Levoy} 
~\cite{Curless1996}, the Truncated Signed Distance Function (TSDF), allowing for reconstruction 
at scene scale. For pose estimation, a multi-level variant of the ICP algorithm utilising a 
point-to-plane metric similar to that of \textit{PL-ICP}~\cite{Censi2008} is used. The pipeline 
consists of four phases; \textit{measurement}, \textit{integration}, \textit{isosurface extraction} 
and \textit{pose update}. Applications of the presented system however are limited only to 
those that require the reconstruction of static scenes; dynamic scenes are not supported 
by \textit{KinectFusion}.

Further optimisations were made in 2013 by \textit{Nei{\ss}ner et al}~\cite{NieBner2013} to the 
\textit{KinectFusion} pipeline proposed by \textit{Newcombe et al}~\cite{Newcombe2011}. The authors 
introduce a spatially hashed TSDF data structure, in which the TSDF is split in to hashed blocks of 
voxels allowing for very fast voxel lookups. The presented approach yields low space and time complexity 
for such operations, vastly increasing the potential for real time, large scale use. Additionally, 
a streaming system is introduced to dynamically handle data transmission between the Central Processing Unit 
(CPU) and GPU, allowing for the reconstruction of scenes that may exceed the Graphical Random Access Memory (GRAM) 
bounds of commodity GPU's. The proposed system is capable of running at \(\approx46Hz\) on an NVIDIA Titan GPU.\@

In the same year, \textit{Thomas et al}~\cite{Thomas2013} introduced an alternative scene 
representation, based on the notion that a scene may be represented as a set of planar 
components with attributes such as surface normal vectors, confidences and Red-Green-Blue (RGB) colour. 
The motivation of the authors approach is that many common scenes that one might reconstruct are indoors and 
consist of components that are planar in nature, such as walls, floors and ceilings. Additionally, 
many planar objects are common, such as tables and cabinets. The authors present an alternative 
rendering approach based on quadrangulation~\cite{Dong2006} and utilise a \textit{KinectFusion} 
~\cite{Newcombe2011} like ICP based algorithm for pose estimation.

\textit{Salas-Moreno et al}~\cite{Salas-Moreno2013} in 2013 also, introduced an alternative 
approach to that of the \textit{KinectFusion}~\cite{Newcombe2011} like pipelines that, similarly to 
\textit{Thomas et al}~\cite{Thomas2013}, utilises the prior information that many scenes consist 
of predictable, repeated structures. As such, the authors introduce a so called ``Object Oriented'' 
dense SLAM paradigm, in which the reconstruction of the scene is split in to a graph of observed 
objects. Pose estimation is achieved by running an ICP based algorithm against renderings of the 
individual objects in the reconstructed scene model. Following pose estimation, the proposed 
system detects newly observed objects and inserts the appropriate object model in to the scene 
model. Consistency between scene components is enforced with pose graph optimisation, with 
re-localisation achieved in a similar manner. The proposed approach does however require a 
database of known objects a-priori.

\textit{St{\"u}ckler et al}~\cite{Stuckler2014} in 2014 introduced a non implicit, non volumetric 
representation based on multiple resolution Surfel~\cite{Pfister2000} maps. The core data structure 
used for scene representation is a Voxel Octree~\cite{Laine2010}, containing both Surfel's and 
probability distributions over appearance and shape. Pose estimation is achieved by optimising for 
a unit quaternion~\cite{Mukundan2002} and translation vector within a maximum likelihood framework, 
in which the energy function to be maximised is the likelihood of the RGBD observations given the 
accumulated probability distributions stored in the Octree. The presented pipeline also incorporates 
a randomised, graph and keyframe based loop closure component.

Following the approach of \textit{Thomas et al}~\cite{Thomas2013}, \textit{Salas-Moreno et al} 
~\cite{Salas-Moreno2014} in 2014 introduced another reconstruction system that utilises the
planarity property of many common scenes. The proposed approach focuses on the detection and 
modelling of planes in the scene, proceeding with their refinement over time. 
The proposed approach generates Surfel~\cite{Pfister2000} Maps from observed RGBD frames, from 
which the planar regions are detected and integrated, filling holes in the reconstruction over time.
The authors utilise an ICP algorithm to register the vertex maps of the RGBD observations and the 
reconstructed model. Additionally, re-localisation is achieved by the use of fern encoding 
~\cite{Glocker2014} on key-frames. 

\textit{Prisacariu et al}~\cite{Prisacariu2014, Kahler2015} in 2014 followed up the optimisations 
to the \textit{KinectFusion} pipeline proposed by \textit{Nei{\ss}ner et al}~\cite{NieBner2013}. 
The authors presented, in addition to the original publication, a technical report and an open 
source implementation. The proposed work provides further improvements to those of 
\textit{Nei{\ss}ner et al}~\cite{NieBner2013} including a number of low level optimisations to the 
core hashed TSDF data structure, it's allocation and update (integration of observation points) and 
the rendering phase of the pipeline. In addition, the authors demonstrate that pose estimation quality 
may be greatly improved by the use of commodity Inertial Motion Unit devices, commonly found 
on mobile phones and tablet computers. \textit{Prisacariu et al} report running times of 
\(\approx47Hz\) on an NVIDIA Shield tablet and \(\approx910Hz\) with a commodity NVIDIA Titan X GPU.\@

\textit{Whelan et al}~\cite{Whelan2015} in 2015 proposed another \textit{KinectFusion} 
~\cite{Newcombe2011} like pipeline intended to enable reconstruction of large scale scenes, 
achieving reconstruction over hundreds of metres. The approach taken by the authors to enable 
such large scale reconstructions is centred around the use of a cyclic buffer on the GPU.\@ For pose 
estimation, the authors impose both geometric and photometric constraints on the camera pose. 
Additionally, the author's approach performs map updates in an as-rigid-as-possible~\cite{Igarashi2005} 
manner, combining frame recognition such that on a recognition event, a map update is performed.
The proposed pipeline provides loop closure capabilities by utilising pose graph optimisation 
~\cite{Grisetti2010}.

\textit{Zhou et al}~\cite{Zhou2015} also, in 2015, proposed another variant of the 
\textit{KinectFusion}~\cite{Newcombe2011} pipeline proposed by \textit{Newcombe et al}.
The authors present improvements to the pose estimation phase of the pipeline, utilising 
contour cues to aid association and enforcing correspondence constraints on the estimated 
pose, with respect to scene geometry. Central to the presented approach is the depth image 
pre-processing steps of in-painting~\cite{Bertalmio2000} regions of the depth image for which 
there are no depth measurements, followed by the aforementioned contour extraction stage.

The optimised pipeline proposed in 2014 by \textit{Prisacariu et al}~\cite{Prisacariu2014} was 
in 2016 improved with the addition of loop closure handling by \textit{Kahler et al}~\cite{Kahler2016}; 
\textit{Kahler} being one of the authors of the original 2014 contribution. Drift correction is achieved 
by the use of a multiple scene representation, with online alignment being performed periodically 
between the scenes. Corrections between the scenes are made via the use of Pose Graph Optimisation 
~\cite{Grisetti2010}. Loop closures are detected by the use of fern conservatories~\cite{Glocker2014} as 
with the contributions of \textit{Salas-Moreno et al}~\cite{Salas-Moreno2013}.

Many of the approaches and techniques evaluated in this section are foundational to the algorithms 
presented in later chapters. The fundamental geometric representations used in Chapters~\ref{chap:moseg},
~\ref{chap:probobj} and~\ref{chap:spp} are variants of the volumetric representation introduced by 
\textit{Curless \& Levoy}~\cite{Curless1996}; the SDF\@. Additionally, a central theme 
in the work that follows is pose estimation, specifically utilising variants of the ICP algorithm, as 
introduced by \textit{Besl \& McKay}~\cite{Besl1992}. Later work on the \textit{KinectFusion} 
pipeline and it's variants~\cite{Prisacariu2014,NieBner2013}, first introduced by \textit{Newcombe et al}
~\cite{Newcombe2011} builds on the aforementioned volumetric representation and pose estimation approaches to 
provide a full, modern pipeline for dense reconstruction. This pipeline, in turn, is foundational to the 
approaches taken in Chapters~\ref{chap:moseg} and~\ref{chap:probobj}.

%% Semantic SLAM.
\section{Semantic SLAM}
~\label{sec:lit_review_semantic}
Over the years there has been much interest within the computer vision research community on 
the semantic understanding of our environment. The ability of machines to recognise and extract 
information about their environments and the components of them (such as people and objects) has 
wide application potential, ranging from autonomous robotics to AR video games. 
The application potential of this semantic scene understanding ability is amplified when it is 
combined with the vast progress that has been made in dense SLAM.\@ This section shall provide a 
survey of research that amalgamates the two fields of semantic scene understanding and SLAM.\@

%Bengio et.al.~\cite{Bengio2013} - representation learning.
%Girshick et.al~\cite{Girshick2014} - feature hierarchies.
\textit{Civera et al}~\cite{Civera2011} in 2011 introduced an approach to semantic SLAM 
that utilises image based features to attach semantic meaning to 3D observations. The SLAM 
system itself is based on Monocular Extended Kalman Filter SLAM~\cite{Smith1990}, with 
semantics added to points via correspondences between Speeded Up Robust Features~\cite{Bay2006}, 
extracted from the observed RGB frames and precomputed object descriptors. Consistency 
is then enforced by a geometric compatibility measure.

\textit{St{\"u}ckler et al}~\cite{Stuckler2012} in 2012 presented a semantic dense SLAM 
pipeline for the object centric integration of RGBD images. Given an RGBD frame, objects 
are detected using a Random Forest (RF)~\cite{Ho1995} classifier trained on hand crafted features 
extracted from RGBD images. The proposed approach does not reconstruct an entire scene, rather it 
reconstructs scene components (such as objects) that have been semantically segmented 
from the current RGBD frame.

\textit{Valentin et al}~\cite{Valentin2015} in 2015 proposed a fully integrated dense SLAM 
and semantic scene understanding pipeline with interaction being a primary focus. The 
proposed pipeline at it's core is based on that of \textit{KinectFusion}~\cite{Newcombe2011}, 
so requires the use of RGBD images and is restricted to the reconstruction of static scenes. 
Once a scene has been reconstructed, the author's pipeline allows users to interact with 
objects in the scene to provide training data for streaming RFs~\cite{Abdulsalam2007}, 
which are used to detect and label parts of the rendered isosurface belonging to a given 
object class. Segmentations are refined using Variational Bayesian Mean Field Inference 
~\cite{Xing2002, Krahenbuhl2011}. The features extracted for this training 
process are Voxel Oriented Patch features, consisting of surface normal vectors
and appearance information using the CIELab colour space.

\textit{Golodetz et al}~\cite{Golodetz2015}, in the same year, released an open source 
implementation of the pipeline proposed by \textit{Valentin et al}~\cite{Valentin2015}, 
utilising the implementation of the \textit{KinectFusion}~\cite{Newcombe2011} pipeline 
provided by \textit{Prisacariu et al}~\cite{Prisacariu2014}. The framework proposed by 
the authors extends that of \textit{Valentin et al}~\cite{Valentin2015} greatly, for 
example by supporting the use of motion capture systems and VR headsets. 
In addition, the implementation provided is optimised to allow for real time use.

\textit{Handa et al}~\cite{Handa2015}, again in 2015 introduced an alternative, 
real time dense semantic SLAM pipeline. Much like the approaches of \textit{Valentin et al} 
~\cite{Valentin2015} and \textit{Golodetz et al}~\cite{Golodetz2015}, the proposed system 
is based on the \textit{KinectFusion}~\cite{Newcombe2011} pipeline, with semantic scene 
understanding performed on the rendered isosurface. Contrary to previous approaches however, 
the authors make use of stacked Deep Autoencoders~\cite{Liou2008}, trained on synthetic depth 
images a priori. As such, the proposed system makes use only of depth cues and may not be 
adapted to new object classes on an ad-hoc basis.

\textit{Cavallari et al}~\cite{Cavallari2016}, in the following year, presented another 
semantic dense SLAM pipeline built on top of the dense SLAM system presented by 
\textit{Nei{\ss}ner et al}~\cite{NieBner2013}. Much like the work of \textit{Handa et al} 
~\cite{Handa2015}, the proposed approach depends on a model pre-trained on a set of object 
classes. Unlike \textit{Handa et al}~\cite{Handa2015}, the authors make use of an 
Fully Connected Network~\cite{Long2015}, taking the Probability Mass Function (PMF) 
output to determine the class to be assigned to an isosurface region.

Later work by \textit{McCormack et al}~\cite{McCormac2017} in 2017, further integrates CNN based 
semantic information with dense SLAM.\@ The authors make use of CNN generated semantic probability 
maps over a set of object classes when densely reconstructing a scene, using the approach of 
\textit{Whelan et al}~\cite{Whelan2016}. The instantaneous, pixel-wise distributions over class labels 
are fused into the scene model via the use of a Bayesian update procedure. The authors report a high 
degree of semantic accuracy with their approach, and highlight that due to the multi-view nature of the 
approach, an improvement on the 2D case is also achieved on the NYUv2 dataset.

Following thir aforementioned \textit{2017} contribution, \textit{McCormack et al}~\cite{McCormac2018} 
introduce an object centric Dense SLAM system, leveraging recent advances in 2D semantic segmentation 
understanding~\cite{He2017} to augment the traditional Dense SLAM pipeline. The authors utilise 2D 
segmentation masks to spawn object centric TSDF volumes, into which RGBD depth measurements are fused. 
Contrary to the traditional point-to-plane, ICP based tracking (and it's variants), the authors estimate 
pose over a graph of individual objects, including the handling of loop closure events. Spurious object 
instance detections are suppressed by maintaining an existence probability for each detection and it's 
corresponding TSDF volume.

The theme of dense 3D reconstruction with semantics is directly related to the research objectives 
outlined in Section~\ref{sec:intro_aims_structure}. As outlined in this section, little research has 
investigated semantic learning \textit{directly} on 3D geometry, though approaches such as that of 
\textit{Handa et al}~\cite{Handa2015} work towards this. Prominent, prohibitive factors include the 
lack of readily available 3D data and the increased complexity of dealing with environments in which 
such capabilities are desirable, as outlined in Chapter~\ref{chap:intro}. The contributions that follow 
in this work are intended to facilitate such research.

%% Dynamic & Non-Rigid SLAM.
\section{Dynamic & Non-Rigid SLAM, Motion Segmentation and Optical Flow}
~\label{sec:lit_review_dynamic}
Sections~\ref{sec:lit_review_tam} and~\ref{sec:lit_review_semantic} provided an assessment 
of pertinent literature in the fields of SLAM and Semantic SLAM.\@ However, all of the 
approaches outlined in these sections are limited to use in static scenes only without 
the capability to accurately operate in an environment that contains moving or deforming objects. 
This section shall explore pertinent literature on the topics of \textit{Dynamic SLAM}, 
\textit{Motion Segmentation} and \textit{Optical Flow}. As such, the general focus of the 
work surveyed in this section is the detection, estimation and segmentation of motion in 
dynamic scenes.

\textit{Tsap et al}~\cite{Tsap2000}, in 2000, presented an algorithm for non-rigid motion 
tracking of objects. The presented approach solves for dense motion vector fields between 
3D objects by modelling motion with finite elements. The proposed system analyses differences 
between actual and predicted behaviour, using gradient descent to find a set of optimal parameters 
for the non-linear Finite Element Model. Additionally, pose estimation is improved by using point 
correspondences.

\textit{Chen et al}~\cite{Chen2011}, in 2011, introduced a system to perform non-rigid motion 
tracking of the human body. The proposed system extracts and skins a surface mesh from 
multi-view video, after being fitted with a skeleton prior. To solve for non-rigid, articulated 
motion, the authors utilise a weighted, hierarchical ICP algorithm, where weightings are obtained 
by the Approximate Nearest Neighbour~\cite{Indyk2000} algorithm.

In the following year, \textit{Sun et al}~\cite{Sun2012} proposed an approach to motion estimation 
for objects in images. The proposed approach estimates optical flow in a layered manner, where each 
layer pertains to an object undergoing rigid body motion, with the number of layers being determined 
automatically. The authors utilise Maximum Flow~\cite{Lamich2017} to solve a discretised flow field cost 
function for each layer, where object layers are a set of depth ordered Markov Random Fields (MRF) 
~\cite{BishopPRML, Murphy2012ML}.

\textit{Unger et al}~\cite{Unger2012}, again in 2012, proposed an alternative system for the 
estimation of motion of objects undergoing rigid body motion in images. The authors present a 
variational formulation for motion estimation and segmentation with occlusion handling. As with 
the contributions of \textit{Sun et al}~\cite{Sun2012}, the authors utilise a parametric labelling 
of the flow field for each object undergoing motion, with labels encoded with an MRF Potts Model
~\cite{Levada2008}. However, contrary to \textit{Sun et al}~\cite{Sun2012} who utilise a Maximum Flow 
algorithm over the MRF models, \textit{Unger et al} solved for flow and labels within a Primal-Dual
~\cite{Boyd2004Convex} based optimisation framework.

In the same year, \textit{Herbst et al}~\cite{Herbst2013} proposed an extension to optical 
flow estimation to 3D scenes; the proposed system solves for Scene Flow based on RGBD data. 
The proposed approach is similar to that of \textit{Brox et al}~\cite{Brox2004}, with scene flow 
being formulated as a variational optimisation problem. The presented approach is a generalisation 
of the well established variational optical flow algorithm of \textit{Brox et al}~\cite{Brox2004}.

In the following year, \textit{St{\"u}ckler et al}~\cite{Stueckler2013} presented a framework 
for the segmentation of rigid body motion from RGBD data. The authors represent regions undergoing 
rigid body motion and their associated motion parameters as latent variables, with the resultant 
segmentations and parameters being solved for within an Expectation Maximisation
~\cite{BishopPRML, Murphy2012ML} framework. The presented approach is robust to both simultaneous 
foreground and background motion by giving each parity in the probabilistic model.

Though the motion estimation and segmentation approaches reviewed up to this point have not 
been within the SLAM framework, \textit{Keller et al}~\cite{Keller2013} in 2013 introduced an 
RGBD based dense SLAM system capable of segmenting motion in a reconstructed scene. Unlike the 
\textit{KinectFusion}~\cite{Newcombe2011} inspired dense SLAM pipelines, the presented approach 
does not utilise an implicit, volumetric representation. Rather, the authors opt for an explicit 
Surfel~\cite{Pfister2000} based representation. Whilst performing live reconstruction, the proposed 
system detects and uses ICP outliers to determine dynamic scene components. With the information gained 
from detecting ICP outliers, the proposed system then propagates these detections by a flood fill operation. 
As the proposed approach utilises an explicit, flat data structure for scene representation, it does not 
have the advantages of it's highly optimised volumetric counterparts, such as that proposed by 
\textit{Prisacariu et al}~\cite{Prisacariu2011}. As such, scalability is limited.

In 2015, \textit{Perera et al}~\cite{Perera2015} presented an approach to motion segmentation in 
TSDF volumes. Similar to it's planar counterparts presented by \textit{Sun et al}~\cite{Sun2012} 
and \textit{Unger et al}~\cite{Unger2012}, the authors utilised a Markov network over the domain 
of interest. Motion segmentation is posed as a Maximum a Posteriori (MAP)
~\cite{BishopPRML, Murphy2012ML} inference problem over a Conditional Random Field (CRF)
~\cite{Krahenbuhl2011} defined over TSDF voxels. The proposed system is able to segment objects 
undergoing both minor and major displacements, with motion labels and parameters found with 
respect to the live frame and the TSDF\@ However, the proposed approach is limited only to very 
small scenes, with very long running times reported for TSDF volumes of dimensionality 
\( 256 \times 256 \times 256\).

\textit{Newcombe et al}~\cite{Newcombe2015} again in 2015, introduced a dynamic dense SLAM 
system based on the earlier \textit{KinectFusion}~\cite{Newcombe2015} pipeline, with the addition 
of the ability to handle non-rigidly deforming scenes. Non-rigid deformations are handled by the 
estimation of a 6DoF motion field that warps the model represented by the TSDF to the live frame. The 
solving of the warp field is achieved by the use of Dual Quaternion blending~\cite{Kavan2006}. 
Though promising results are presented, there are limitations, such as lack of robustness to 
open/closed topology changes, such as hands. In addition, the authors highlight scalability issues.

The research outlined in this section is pertinent to the work presented in Chapter~\ref{chap:moseg}, 
which presents an approach to the handling of dynamics for dense reconstruction in real time, and 
with a volumetric representation. Though the aforementioned works of \textit{Perera et al}
~\cite{Perera2015} and \textit{Newcombe et al} provide algorithms for handling dynamics in such 
representations, there remain complexity and scalability issues, the target of which is the focus of 
Chapter~\ref{chap:moseg}.

%% Object Reconstruction.
\section{Object Reconstruction}
~\label{sec:lit_review_obj_recon}
It is evident from Sections~\ref{sec:lit_review_tam},~\ref{sec:lit_review_dynamic} and 
~\ref{sec:lit_review_semantic} that much progress has been made in the fields of TAM/SLAM, 
dynamic SLAM and semantic SLAM.\@ However, \textit{Object Reconstruction} remains a very open 
and active field of research. As outlined in Section~\ref{sec:lit_review_obj_recon}, cumulative 
errors in pose estimation are troublesome for the smaller scale (relative to scene-scale), 
object centric SLAM\@. The combination of inherently less geometric information and potentially 
rapid, repetitive motion exacerbates the difficulties faced in scene-scale SLAM\@.

This section provides a review of pertinent literature on the task of reconstructing consistent 
models of objects, rather than full scale scenes. The problem of interest in this section, though 
related to SLAM, incurs additional complications with regards to pose estimation.

\textit{Curless \& Levoy}~\cite{Curless1996} as introduced in Section~\ref{sec:lit_review_tam} 
presented a method of statically reconstructing shapes from range images taken from different 
viewpoints. However, the presented approach pre-dates many of the advances that have allowed 
for simultaneous tracking and mapping.

\textit{Kolev et al}~\cite{Kolev2006}, in 2006, presented a probabilistic approach to 3D shape 
segmentation and recovery. Rather than the direct reconstruction approach taken by 
\textit{Curless \& Levoy}~\cite{Curless1996}, the authors take the approach of inferring the most 
probable shape with respect to the observed image sequences. The shape to be inferred is encoded as a 
zero level set, extracted from a level set representation (such as an SDF). The level set of the shape is 
evolved over time within a variational framework, with respect to a volume of segmentation 
probabilities (foreground versus background). However, the proposed approach does not have an additional pose 
estimation phase and has only been evaluated on synthetic data of very polarised appearance.

\textit{Weise et al}~\cite{Weise2009}, in 2009, proposed an approach to the in hand scanning of 3D objects. 
The authors utilise an explicit point cloud representation of shape, rendered as Surfels~\cite{Pfister2000}. 
Objects are rotated in front of a sensor with poses recovered by the use of an ICP like algorithm. During 
pose estimation, a topology graph is built which is used to offset drift in estimated poses in an 
as-rigid-as-possible~\cite{Igarashi2005} manner. However, the specification of object rotation in front of 
a sensor is suggestive of limited tracking ability. In addition, the type of sensor is not specified, 
as such it is not clear what quality of sensing equipment is required to yield high quality results.

\textit{Llado et al}~\cite{Llado2011} introduced in 2011 an approach to the 3D reconstruction of 
\texit{Deformable} objects. The proposed approach makes use of an uncalibrated RGB stereo rig, requiring 
minimal \textit{a-priori} setup. The approach to nonrigid reconstruction taken centers around the computation 
of a mean shape, to which live and reference frames are registered. From the registration to the mean shape, 
rigidly moving points may be identified. The final \textit{deformed} model is recovered as a nonlinear 
optimisation problem. 

\textit{Prisacariu et al}~\cite{Prisacariu2012} in \textit{2012} proposed a probabilistic approach to the 
simultaneous tracking and segmentation of objects with \textit{a priori known} 3D shape. Appearance based 
segmentation is performed in 2D, utilising Pixel Wise Posteriors (PWP)~\cite{Bibby2008}, with tracking performed 
in 3D. The authors demonstrate real time performance with the use of GPU hardware and propose a simple extension 
to the multiple object tracking case.

In 2013, \textit{Garg et al}~\cite{Garg2013} introduced an approach to dense, ron-rigid surface reconstruction 
from monocular video. The authors present an approach to non-rigid SfM as a variational energy minimisation 
problem that does not require a shape prior. The author's report efficacy on the modelling of an instantaneous 
objects deformed state for a given frame.

Also in 2013, \textit{Ren et al}~\cite{Ren2013} proposed an approach to the tracking and reconstruction of 
objects. Like \textit{Kolev et al}~\cite{Kolev2006}, the authors utilise a probabilistic formulation 
based on the evolution of a level set representation. Initialised with a shape prior level set, the 
proposed approach evolves the shape prior with respect to observations. Crucially, unlike the approach of 
\textit{Kolev et al}~\cite{Kolev2006}, the proposed approach simultaneously optimises for object pose.
The proposed system works with RGBD data and segments the object of interest using PWP~\cite{Bibby2008}, as 
with the aforementioned work of \textit{Prisacariu et al}~\cite{Prisacariu2012}. It is noteworthy however 
that there are performance limitations of the proposed approach and experiments show success for a limited 
set of target shapes.

\textit{Agudo et al}~\cite{Agudo2014} in 2014 proposed an approach to dense, non-rigid SfM. The authors outline 
an algorithm that is sequential in nature (rather than offline, batch processed) and takes as input a single 
monocular RGB stream. The approach taken is to model the mechanical dynamic behaviour of an objects surface over 
a rolling temporal window. The authors perform EM in a tractable manner by marginalising over the time 
dependent deformation parameters of the mechanical model. The proposed approach demonstrates efficacy for 
on-line, instantaneous recovery of a given objects mesh.

Later in 2015, \textit{Dou et al}~\cite{Dou2015} present a system for the reconstruction of deformable 
objects using a Microsoft Kinect RGBD sensor. The proposed approach solves for a latent target shape 
and shape deformations by utilising bundle adjustment~\cite{Triggs1999}. The authors report that loop 
closures are automatically detected, with errors incurred by drift being distributed backwards from the 
detection point. The resultant shape surface is extracted as a triangular mesh. The presented experiments 
demonstrate high quality reconstruction results, but with overnight run times indicating that it is not 
suitable for real time use.

Also in 2015, \texit{Yu et al}~\cite{Yu2015} proposed a novel approach to non-rigid SfM using a monocular 
RGB video stream. The authors outline a shape template based algorithm, in which the template shape is 
constructed from a short rigid sequence (i.e an object of interest is not \textit{non-rigidly deforming}). 
Once a template shape has been obtained, the algorithm proceeds to recover, for each frame, the deformed 
object model via an energy minimization over a photometric loss. Though the authors present impressive, 
high quality results, the requirement of a rigid sequence \textit{a-priori} may limit the scope of 
applicability of the approach.

In the following year, \textit{Gupta et al}~\cite{Gupta2016} proposed a system for the reconstruction and 
segmentation of 3D objects from data obtained with an RGBD sensor. The reconstructed object is represented 
implicitly within an SDF volume, but notably observations are integrated using the Softmax~\cite{Murphy2012ML} 
function rather than weighted means as with \textit{KinectFusion}~\cite{Newcombe2011}. Each voxel in the 
volume is assigned a label pertaining to it's membership of the object set, with objects refined utilising 
Graph Cuts~\cite{CLRS} and Alpha Expansions~\cite{CLRS}. The proposed approach utilises a photometric loss to optimise 
for object pose, with keyframe based loop closure detection. However, the authors report difficulties in 
building sufficiently granular reconstructions. In addition, the authors report drift in pose estimation 
to be problematic.

The contributions of Chapter~\ref{chap:probobj} for the problem of object reconstruction outlined in 
Section~\ref{sec:intro_object_recon} draw on the work of \textit{Kolev et al}~\cite{Kolev2006}. The 
probabilistic volumetric representation used for the authors level set evolution approach influences the 
formulation of object segmentation given in Chapter~\ref{chap:probobj}. Additionally, the work of 
\textit{Ren et al}~\cite{Ren2013} provides a suitable base of comparison for the approach outlined later 
in this work. The approach of \textit{Ren et al} performs simultaneous segmentation, pose estimation and 
reconstruction of objects from an RGBD image source, as is the case with the approach taken in Chapter
~\ref{chap:moseg}.

%% Object Shape Prediction.
\section{Shape and Pose Prediction}
~\label{sec:lit_review_prediction}
Section~\ref{sec:lit_review_obj_recon} provided a review of pertinent object reconstruction research, 
which demonstrates that although much progress has been made since the early work of \textit{Curless \& 
Levoy}~\cite{Curless1996}, many open research problems remain. This section provides a survey of research 
into an alternative, inference driven approach to obtaining 3D models of observed 
objects, whereby rather than direct optimisation and integration being used for pose estimation and 
model building, the process is posed as a probabilistic inference procedure.

\textit{Prisacariu et al}~\cite{Prisacariu2011} in 2011 introduced an approach to shape prediction, 
segmentation and pose estimation. Shape is predicted from a hierarchy of generative Gaussian Process 
Latent Variable Models (GPLVM)~\cite{Lawrence2005}, encoding a latent space embedding of common shape 
properties. Candidate shapes are generated as a one off regression in latent space, with a unified 
energy function optimised with respect to the shape latent space point and the object pose parameters.

In 2013, \textit{Dame et al}~\cite{Dame2013} proposed an approach to dense object reconstruction from a 
monocular image source. Like the approach of \textit{Prisacariu et al}~\cite{Prisacariu2011}, the authors 
utilise GPLVM's as shape priors to aid reconstruction and segmentation of the object of interest. Depth maps 
for the observed monocular sequence are optimised for within a Primal-Dual~\cite{Boyd2004Convex} framework, utilising 
Total Variation~\cite{Rudin1992} regularisation. However, there is no pose estimation 
ability in the formulation, as poses are known a priori from PTAM~\cite{Klein2007}.

In the following year, \textit{Toshev et al}~\cite{Toshev2014} proposed an approach to pose estimation 
utilising cascaded Deep Neural Network (DNN)~\cite{LeCun2015} regressors. The authors utilise the DNN 
framework for the complex task of articulated human pose estimation. %TODO: elaborate

\textit{Wohlhart et al}~\cite{Wohlhart2015}, in 2015, presented an approach to the 3D detection and pose 
recovery of objects. The proposed approach utilises features extracted from a Convolutional Neural 
Network (CNN)~\cite{LeCun2015} within a nearest neighbour cost function for object detection and recovery 
of rough pose. As such, the proposed approach poses the problem as a K-Nearest Neighbour
~\cite{Altman1992} search in descriptor space. Object and pose are coupled in training (i.e.\ two similar 
cars with different poses will have spatially distant descriptors).

\textit{Chang et al}~\cite{Chang2015}, also in 2015, presented a large scale dataset of 3D shapes. The 
dataset can be used for a variety of 3D vision tasks due to the potential of modern machine learning 
techniques to learn rich latent space embeddings, as demonstrated by the approaches of 
\textit{Prisacariu et al}~\cite{Prisacariu2011} and \textit{Dame et al}~\cite{Dame2013}. However, the 
shapes in the dataset are synthetic and as such have no depth sequences. Though, such sequences may be 
artificially rendered.

Also proposed in 2015 by \textit{Rock et al}~\cite{Rock2015}, is an approach to the recovery of complete 
3D models from a single depth image of an object of interest. The input depth image is regressed into a 
database of a priori known objects by the use of an RF~\cite{Ho1995}. The matched shapes are 
coarsely matched to the input depth map, then later deformed at a higher granularity by a separate 
optimisation process.

\textit{Kendall et al}~\cite{Kendall2015}, again in 2015, proposed a CNN approach to the regression of 
6DoF camera pose from RGB input. The authors base their CNN architecture on that of \textit{Szegedy et al}
~\cite{Szegedy2014}, with a depth of 23 layers. The authors report high levels of accuracy on indoor scenes, 
attributing this to the use of Transfer Learning~\cite{Pan2010} applied to classification models.

In 2017, \textit{Zhou et al}~\cite{Zhou2017} introduced an approach to object detection from 3D point 
clouds. Central to the proposed approach is an end-to-end trainable convolutional Region Proposal 
Network~\cite{Girshick2015_2}. The authors evaluate the proposed approach on the KITTI LIDAR
~\cite{Geiger2013} dataset, with input point clouds quantized into a voxel volume prior to training 
and prediction.

\textit{Gwak et al}~\cite{Gwak2017}, also in 2017 introduced a Generalised Adversarial Network
~\cite{Goodfellow2014} like approach to shape prediction. The proposed network is trained in a 
weakly-supervised manner on silhouettes and 3D shapes with a log-barrier objective function. 
However, the applicability to ``real world'' scenarios is questionable, due to the synthetic nature of 
the data used to train and evaluate the network.

\testit{Grabner et al}~\cite{Grabner18} in \textit{2018} introduce a CNN based approach to the problem of 
simultaneous pose estimation and 3D shape retrieval. The authors use the estimated pose of a given 
object of interest as a shape prior for 3D model lookup. The authors render depth images of a given retrieved 
shape under the predicted pose for evaluation in a multi-view photometric loss for evaluation against 
learned image descriptors. The authors report impressive performance on \textit{Pascal3D+}~\cite{Xiang2014}.

\textit{Pumarola et al}~\cite{Pumarola2018}, also in \textit{2018} introduced an approach to the prediction 
of deformable shape surfaces from a single view. The outlined algorithm takes a two stage, CNN based approach 
consisting of detection and shape estimation, respectively. The approach is evaluated a synthetic dataset to 
which the authors artificially apply deformations and varying textures. The proposed approach demonstrates 
efficacy in both synthetic and non-synthetic data scenarios.

\section{Summary}
~\label{sec:lit_review_summary}
From the evaluation of the literature given in Sections~\ref{sec:lit_review_tam},~\ref{sec:lit_review_dynamic},
~\ref{sec:lit_review_semantic},~\ref{sec:lit_review_obj_recon} and~\ref{sec:lit_review_prediction}, it is clear 
that much progress has been made in many areas of 3D computer vision, with a high level of commonality across the different 
domains. However, when assessed within the context of the research objectives of this work, as outlined in Section
~\ref{sec:intro_aims_structure}, there is still much that can be contributed.

The literature reviewed in Sections~\ref{sec:lit_review_tam} and~\ref{sec:lit_review_dynamic} is directly related to 
the subject matter of Chapter~\ref{chap:moseg}, which approaches the problems faced when utilising dense SLAM techniques 
in an environment that is dynamic, rather than static, as with the approaches outlined in Section~\ref{sec:lit_review_tam}.
Though it is evident from the literature assessment of Section~\ref{sec:lit_review_dynamic} that much progress has been made, 
it is also evident from Section~\ref{sec:lit_review_dynamic} that there exist limitations in current work. 
Section~\ref{sec:moseg_introduction} introduces approaches to solving some of these limitations.

Section~\ref{sec:lit_review_obj_recon} reviewed literature pertinent to the object reconstruction subject matter of 
Chapter~\ref{chap:probobj}. Though it is clear that many advances have been made on representations and combined TAM for 
objects, Section~\ref{sec:probobj_introduction} introduces an approach to the ongoing research problem around global model 
consistency. Additionally, the primarily 2D featured nature of the systems outlined in Section~\ref{sec:lit_review_semantic} 
provide a motivation for such approaches, as outlined in Sections~\ref{sec:intro_object_recon} and~\ref{sec:probobj_introduction}.

The data driven works outlined in Section~\ref{sec:lit_review_prediction} provide a basis for the approach to shape and pose 
prediction presented in Chapter~\ref{chap:spp}. Though much of the work reviewed in Section~\ref{sec:lit_review_prediction} 
addresses each of these problems in a \textit{decoupled} way, may of the techniques are pertinent to the integrated approach 
taken in Chapter~\ref{chap:spp}, as outlined in Section~\ref{sec:spp_introduction}.
