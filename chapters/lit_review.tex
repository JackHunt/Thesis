%% Tracking and Mapping.
\section{Tracking and Mapping}
\label{sec:lit_review_tam}
Besl and McKay \cite{Besl1992}
\begin{itemize}
	\item 3D Shape Registration.
	\item Full 6DoF pose estimation.
	\item Requires shape complexity - geometrically distinctive.
	\item Compute closest point, compute \& apply registration, iterate until MSE low. 
	\item Convergence not guaranteed, determined empirically to be rapid over first few iterations.
\end{itemize}

Curless and Levoy \cite{Curless1996}
\begin{itemize}
	\item Early volumetric reconstruction work.
	\item Integrates aligned range images in to a volume.
	\item Volume introduced is SDF - cumulative and weighted.
	\item Isosurface extracted from SDF - Marching Cubes.
	\item Gaps filled by tesselation.
\end{itemize}

Zhou et.al. \cite{Zhou2008}
\begin{itemize}
	\item Alternative representation to Curless and Levoy.
	\item Massively parallel KD-Tree construction.
	\item Suitable for real-time use.
	\item Parallelism achieved by building tree with BFS.
	\item Example use given for Ray Tracing and Photon Mapping.
\end{itemize}

Censi \cite{Censi2008}
\begin{itemize}
	\item PL-ICP, ICP using point-to-line metric.
	\item Closed form solution in planar case.
	\item Quadratic convergence in finite amount of steps.
	\item New formulation weighted by normal and solved by reducing to quadratic form and introducing Lagrange Multipliers.
	\item Prior to optimisation, trimming procedure used to remove outliers.
\end{itemize}

Newcombe et.al. \cite{Newcombe2011}
\begin{itemize}
	\item Real time mapping of indoor scenes with Kinect sensor.
	\item Invariance to lighting.
	\item Observations fused in to SDF like volume of Curless et al - TSDF
	\item Multilevel ICP(coarse to fine) used to obtain pose.
	\item Measurement->integration->isosurface extraction->pose update.
	\item Limited to static scenes.
\end{itemize}

%TO-DO:fix name
Neissner et.al. \cite{NieBner2013}
\begin{itemize}
	\item Pipeline follows that of KinectFusion.
	\item Introduce a spatial Hashing data structure.
	\item TSDF split in to Voxel Blocks which are hashed.
	\item Low space and time complexity.
	\item Streaming system to reduce GPU memory usage.
\end{itemize}

Thomas et.al. \cite{Thomas2013}
\begin{itemize}
	\item Represent scene as a set of planes with attributes.
	\item Motivated by planar nature of objects such as tables and cabinets.
	\item Attributes are bump(normal) image for geometry encoding, mask image encoding confidence and RGB image.
	\item Rendering by quadrangulation. %Cite this
	\item Tracking as with KF - linearized GICP.
\end{itemize}

Salas-Moreno et.al. \cite{Salas-Moreno2013}
\begin{itemize}
	\item Introduce a new "Object Orientated" Dense SLAM paradigm.
	\item Incorporates prior knowledge that many scenes have repeated structure.
	\item Scene split in to graph of objects. Pose graph optimisation used.
	\item ICP run against object renderings, followed by detection and insertion of objects. Finally graph optimisation.
	\item Graph based re-localisation.
	\item Requires database of known objects.
\end{itemize}

%TO-DO:fix name
Stuckler et.al. \cite{Stuckler2014}
\begin{itemize}
	\item Uses multiple resolution, probabilistic surfel maps as representation.
	\item Octree with spatial and appearance statistics at each level.
	\item Randomised loop closure - graph and key-view based.
	\item Pose estimation by maximising observation likelihood with uncertainty measure.
\end{itemize}

Salas-Moreno et.al. \cite{Salas-Moreno2014}
\begin{itemize}
	\item Exploits planar structure in the scene, like Thomas.
	\item Focus on detection and modelling of planes, refined over time.
	\item From generated Surfel maps, planes are segmented and holes filled over time. %Reference Surfels.
	\item Fern encoding used for relocalisation. %Reference Fern Encoding - Glocker et al
	\item ICP between measured vertex map and predicted vertex map.
\end{itemize}

Prisacariu et.al \cite{Prisacariu2014} Followed up by tech report \cite{Kahler2015}
\begin{itemize}
	\item Open source implementation of Voxel Hashing.
	\item A number of optimisations to the data structure(allocation \& integration) and raycaster.
	\item IMU data to supplement camera observations for tracking.
	\item 47Hz NVIDIA Shield tablet, 910Hz Titan X.
\end{itemize}

Whelan et.al \cite{Whelan2015}
\begin{itemize}
	\item Like KF but large scale(hundreds of meters), achieved using GPU cyclic buffers.
	\item Geometric and photometric camera pose constraints.
	\item Map updated by frame recognition, as-rigid-as-possible. %Reference as-rigid-as-possible
	\item Pose graph based loop closure.
\end{itemize}

Zhou et.al. \cite{Zhou2015}
\begin{itemize}
	\item Uses contour cues to improve camera tracking.
	\item Contour cues extracted from noisy and incomplete depth images.
	\item Correspondence constraints using scene geometry enforced on pose estimation.
	\item Based on KF pipeline.
	\item Depth image inpainting used before contour extraction.
\end{itemize}

%TO-DO: fix name.
Kahler et.al \cite{Kahler2016} 
\begin{itemize}
	\item Based on KF pipeline and InfiniTAM.
	\item Online submap alignment algorithm for drift correction.
	\item Inter-submap corrections based on graph optimisation.
	\item Loop closure detected using fern conservatories. %cite glocker
\end{itemize}

%% Semantic SLAM.
\section{Semantic SLAM}
\label{sec:lit_review_semantic}

%% Dynamic SLAM.
\section{Dynamic SLAM}
\label{sec:lit_review_dynamic}

%% Object Reconstruction.
\section{Object Reconstruction}
\label{sec:lit_review_obj_recon}

%% Object Shape Prediction.
\section{Shape and Pose Prediction}
\label{sec:lit_review_prediction}