Advances in 3D computer vision have tremendously impacted the way that humans and computers interact,
ranging from immersive, Virtual Reality video games, to industrial robotics. Such applications of 3D 
vision are underpinned by a range of computer vision competencies, including pose estimation, mapping 
and semantic understanding. However, there remain many difficult, open research problems within the field, 
such as many common mapping approaches being restricted to static environments, the difficulty in obtaining 
high quality 3D models of objects, and scene understanding and modelling in large scale environments, such as 
towns or cities.

In this Thesis, there are three central subject matters that are approached. Firstly, an approach to the 
difficult task of dense mapping in dynamic environments is proposed. Central to the approach is a novel 
representation of the observed scene that allows for dynamic components, such as walking people to be 
separately handled from their static counterparts, such as furniture. It is demonstrated in this work that 
the proposed approach yields an improvement in pose estimation accuracy in dynamic scenes, versus it's 
common, static counterpart. Additionally, the proposed approach is capable of exceeding real-time performance, 
and may be leveraged for simple geometric feature extraction for semantics due to it's motion segmentation
capability.

The second central research topic of this Thesis is the high quality reconstruction of 3D objects. 
The approach to which comprises a novel representation and formulation of the problem, such that errors are 
corrected online. The proposed approach is evaluated against a state of the art method, over which an 
improvement in reconstruction is demonstrated. Additionally, the efficacy of the online correction procedure 
is demonstrated versus a vanilla approach. Finally, high levels of geometric accuracy are seen versus an appropriate 
benchmark.

%%% SPP
%%%%%% Preliminary results indicate ambitous task may be tractable
%%%%%% Promising avenue of research yada yada
The third major focus of this work is the simultaneous inference of object shape and pose in large scale, outdoor environments. 
An ambitous approach to regressing shape and pose in a weakly-supervised manner is presented, utilising a combination 
of Convolutional Neural Networks and Gaussian Processes. Early results indicate\dots

The work that is outlined in this Thesis is intended to provide a strong foundation for further research in the area of 
geometrically driven 3D scene understanding. However, immediate applications are evident. The motion segmentation and 
dense mapping approach for dynamic environments allows for larger scale dense mapping in previously prohibitive scenarios, 
and as such has applications in robotics. The object reconstruction work that constitutes the second major focus of the Thesis 
is applicable to the problem of collecting geometrically consistent 3D object data. Finally, the simultaneous inference of 
shape and pose is applicable to modelling scenarios of specific semantic interest, where an entire scene need not be reconstructed. 
The approach preliminarily demonstrates potential for large scale, semi-dense, geometric and semantic mapping.