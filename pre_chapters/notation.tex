\section*{Mathematical Notation}
This preliminary section introduces the mathematical notation used in this work. The following 
table outlines essential notational details, separated into sections on Sets, Fields and Groups, 
Linear Algebra, Sums and Products, Calculus and Probability Theory.
\begin{longtable}{p{.20\textwidth} | p{.80\textwidth}}
  \hline
  Symbolic Form & Meaning \\
  \hline
  \( \mathcal{A} \) & Calligraphic variables indicate mathematical sets (unless otherwise indicated).\\
  \( \{ a, b, c \} \) & A mathematical set consisting of the elements \( a \), \( b \) and \( c \).\\
  \( \{ a | \rho(a) \} \) & A mathematical set whose elements are defined by predicate \( \rho(a) \).\\
  \( a \in \mathcal{A} \) & The element \( a \), in the set \( \mathcal{A} \).\\
  \( f(a) \forall a \in \mathcal{A} \) & \( f(a) \) for all \( a \) in \( \mathcal{A} \).\\
  \( | \mathcal{S} | \) & Cardinality of \( \mathcal{S} \); the number of elements in \( \mathcal{S} \).\\
  \hline
  \( \mathbb{R}^{N} \) & The field of real numbers, of dimension \( N \).\\
  \( \mathbb{SE}(3) \) & The Special Eucledian Group of \( 4 \times 4 \) transform matrices.\\
  \( \mathbb{SO}(3) \) & The Special Orthogonal Group of \( 3 \times 3 \) rotation matrices.\\
  \hline
  \( \bm{A}\), \( \bm{\Theta} \) & Bold, uppercase symbols indicate matrix or tensor quantities.\\
  \( \bm{a}\), \( \bm{\theta} \) & Bold, lowercase symbols indicate vector quantities.\\
  \( \lVert \bm{a} \rVert_{n} \) & \( n \)-norm of a vector.\\
  \( \bm{a}^{T} \), \( \bm{A}^{T} \) & Transpose of a vector or matrix.\\
  \( \bm{A}^{-1} \) & The matrix inverse of \( \bm{A} \).\\
  \hline
  \(\sum_{a = 0}^{N} a\) & The sum from \( 0 \) to \( N \) of \( a \).\\
  \(\sum_{a \in \mathcal{A}}\) & The sum of the elements in \( \mathcal{A} \).\\
  \(\prod_{a = 0}^{N} a\) & The product from \( 0 \) to \( N \) of \( a \).\\
  \hline
  \( \frac{\partial f(\bm{a})}{\partial \bm{a}_{i}} \) & Partial derivative of \( f \) with respect to the \( i^{th} \) element of \( \bm{a} \).\\
  \( \nabla f(\bm{a}) \) & Gradient vector of \( f(\bm{a}) \). The vector of partial derivatives.\\
  %\hline
  % From Volumetric Appearance Model
~\label{table:mathematical_notation}
\end{longtable}

\section*{Abbreviations}